% pacakges
\usepackage{pdflscape}
\usepackage{geometry}
\usepackage{hyperref} %  link
\usepackage{graphicx} %immagini
\usepackage{titlesec} %per creazione di una section personalizzata
\usepackage{float} %per il posizionamento delle figure

\usepackage[shortlabels]{enumitem} % per elenchi personalizzati
\usepackage{titlesec} %per creazione di una section personalizzata

% package per la lingua / caratteri
\usepackage[italian]{babel}
\usepackage[utf8]{inputenc}
\usepackage[T1]{fontenc}

\usepackage{fancyhdr} % per header e footer
\usepackage{lastpage} % per avere l'indice dell'ultima pagina

\usepackage{tabularx} % tabelle
\usepackage[table]{xcolor} % definizione colore di sfondo per tabelle
\usepackage{longtable} % permette di estendere le tabelle su più pagine

\usepackage{chngcntr} % per numerazione immagini e tabelle

\usepackage[official]{eurosym} %per poter utilizzare il simbolo dell'euro

\usepackage[titletoc]{appendix} % per poter usare le appendici
\usepackage[official]{eurosym} %per poter utilizzare il simbolo dell'euro
\usepackage{float} %posizionamento delle immagini UC

%Per la creazione dei grafici
\usepackage{pgfplots}
\usetikzlibrary{
	pgfplots.dateplot,
}
\pgfplotsset{width=15cm,compat=1.11}
\usepackage{tikz}

% configurazione
% link
\hypersetup{
	colorlinks=true,
	linkcolor=black,
	filecolor=magenta,
	urlcolor=blue,
}

% intestazione e piè di pagina
\pagestyle{fancy}
% intestazione
\setlength{\headheight}{25pt}
\lhead{ \includegraphics[scale=0.3]{./res/img/cropped_logo.png} }
\rhead{ \docTitle{} }
% piè di pagina \\
\renewcommand{\footrulewidth}{0.4pt} % per avere una linea nel footer
\cfoot{}
\rfoot{Pagina \thepage{} di \pageref{LastPage}}

% tabelle
\def\arraystretch{1.5} % padding
% comandi e colori tabelle
\definecolor{lightRowColor}{HTML}{fafafa}
\definecolor{darkRowColor}{HTML}{ffcccb}

\newcommand{\coloredTableHead}{\rowcolor[HTML]{b61827}}
\newcommand{\lightTableRow}{\rowcolor{lightRowColor}}
\newcommand{\darkTableRow}{\rowcolor{darkRowColor}}

% per numerazione immagini e tabelle
% --> la numerazione dipende dalla subsection in cui ci si trova
\counterwithin{table}{subsection}
\counterwithin{figure}{subsection}

% definizione del comando subsubsubsection
\titleclass{\subsubsubsection}{straight}[\subsection]

\newcounter{subsubsubsection}[subsubsection]
\renewcommand\thesubsubsubsection{\thesubsubsection.\arabic{subsubsubsection}}

\titleformat{\subsubsubsection}
  {\normalfont\normalsize\bfseries}{\thesubsubsubsection}{1em}{}
\titlespacing*{\subsubsubsection}
{0pt}{3.25ex plus 1ex minus .2ex}{1.5ex plus .2ex}

\makeatletter
\def\toclevel@subsubsubsection{4}
\def\toclevel@paragraph{5}
\def\toclevel@paragraph{6}
\def\l@subsubsubsection{\@dottedtocline{4}{7em}{4em}}
\def\l@paragraph{\@dottedtocline{5}{10em}{5em}}
\def\l@subparagraph{\@dottedtocline{6}{14em}{6em}}
\makeatother

\setcounter{secnumdepth}{4}
\setcounter{tocdepth}{4}
